%===============================================================================
% $Id: ifacconf.tex 19 2011-10-27 09:32:13Z jpuente $  
% Template for IFAC meeting papers
% Copyright (c) 2007-2008 International Federation of Automatic Control
%===============================================================================
\documentclass{absconf}

%\usepackage{graphicx}      % include this line if your document contains figures
\usepackage{natbib}        % required for bibliography
%\usepackage{libertine}
%===============================================================================
\begin{document} \begin{frontmatter}

	\title{Power Allocation and Relay Selection in Amplify-and-Forward
	Relaying}
	% Title, preferably not more than 10 words.

	\author{Prudhvi Porandla, Prof. Prasanna Chaporkar} \address{ Electrical
	Engineering, IIT Bombay   } 
	\vspace{12pt} 
	\begin{abstract}
		 \mbox{} \\ 
		 Multihop
		communication is considered to be a standard in next generation
		cellular networks.  There are several relaying schemes,
		Decode-and-forward(DF) and Amplify-and-Forward(AF) being the popular
		ones. In DF scheme, the relay decodes the message from the source,
		re-encodes and transmits it to the destination node whereas in AF
		the relay amplifies the received signal and transmits to the
		destination node. Relay selection and optimal power allocation are
		the two important aspects in either scheme. In this work, we look at
		these two problems in 2-hop communication network in which relays
		employ AF scheme. To make the power allocation problem well-defined
		we prove that rate/capacity is a concave function of both source and
		relay powers. Once concavity is established, we can find the optimal
		relay and source powers. However, when there are multiple relays the power allocation might interfere with relay selection. We show that this is indeed the case and discuss the conditions under which a relay switch over can take place.  

	\end{abstract}

\begin{keyword} 
	Amplify-and-Forward, relay selection, power control, cooperative communication
\end{keyword}

\end{frontmatter}
%===============================================================================
\end{document}
