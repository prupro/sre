\documentclass[titlepage]{article}

\usepackage{amsmath}  % For math
\usepackage{amssymb}  % For more math
\usepackage{libertine}


\begin{document}
\textbf{Amplify-and-Forward} \\
The transmission has two slots. In th first, source node broadcasts the signal to
relay and destinatio and in the senconf the realy amplifies the received signal 
and retransmits it to the destination node. \\
\textit{First Slot:}
\begin{equation}
	Y_{sd} = \sqrt{P_s G_{sd}} X_s + n_{sd} \\
	Y_{sr} = \sqrt{P_s G_{sr}} X_s + n_{sr} \\
\end{equation}
\textit{Second Slot:}
\begin{equation}
	Y_{rd} = \sqrt{P_r G_{rd}} X_{rd} + n_{rd} \\
	X_{rd} = \frac{Y_{sr}}{|Y_{sr}|} \\
\end{equation}

\begin{equation}
	Y_{rd} = \frac{\sqrt{P_r G_{rd} P_s G_{sr}}}{\sqrt{P_s G_{rd} + \sigma^2}}X_s + \frac{\sqrt{P_rG_{rd}}}{\sqrt{P_s G_{rd} + \sigma^2}} n_{sr} + n_{rd} 
\end{equation}

\textit{Rate expression:}
\begin{equation}
	R = \frac{1}{2} w log_2(1+\Gamma_{sd}+\Gamma_{srd}) \quad where \Gamma 
	represents SNR
\end{equation}

\begin{equation}
R = \frac{1}{2} w log_2(1+\frac{P_s g_{sd}}{\sigma^2} +
\frac{P_s g_{sr} P_r g_{rd}}{\sigma^2(\sigma^2 + P_sg_{sr} + P_rg_{rd})})
\end{equation}

Assume there are two relays and let $a_1 = \frac{g_{sr_1}}{\sigma^2}$
and $b_1 = \frac{P_{r} g_{r_1 d}}{\sigma^2}$ similarily $a_2, b_2$ for relay 2.
At a particular source power $P_s$, relay 1 is chosen over relay 2
if $\frac{a_1b_1}{1+P_s a_1 + b_1} > \frac{a_2b_2}{1+P_s a_2 + b_2}
 $
\\
Consider the function $f(p) = \frac{pab}{1+p a + b} $ \\
\begin{equation}
f'(p) = \frac{(1+b)ab}{(1+pa+b)^2}
\end{equation}
$f'(p)$ is positive and decreasing with $p$. \\
The power at which both the relays give same rate can be obtained 
by equating $f_1(p)$ and $f_2(p)$.

\begin{equation}
	P_0 = (1+b_1)(1+b_2) \frac{\frac{a_1b_1}{1+b_1} - 
	\frac{a_2b_2}{1+b_2}}{a_1a_2(b_2-b_1)}
\end{equation}

For $P_0$ to be positive, both numerator and denominator should 
have same sign i.e., if 
$\frac{a_1b_1}{1+b_1} >	\frac{a_2b_2}{1+b_2}$ then $b_2 > b_1$.
	To explain this intuitively, let us assume $b_0,b_2$ to be 
	much larger than 1 which reduces the first inequialitely ti
	$a_1 > a_2$. What this means is, source to relay cahennel is 
	better for relay 1 but relay to destination channel is stronger
	for relay 2. Hence at low source powers relay 1 might givr
	better SNR but if we increase source power past $P_0$ relay 2 
	gives more rate than relay 1. Same arguments can be made 
	for the case where inequalities are in the opposite
	direction. 
	\\
	For $P_0 < 0$, one of the rel;ays is the desired one irrespec
	tve of source pwer.
\end{document}
