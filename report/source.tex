\documentclass[titlepage]{article}

\usepackage{amsmath}  % For math
\usepackage{amssymb}  % For more math
\usepackage{libertine}


\begin{document}

\begin{equation}
R = \frac{1}{2} w log(1+\frac{P_s g_{sd}}{\sigma^2} +
\frac{P_s g_{sr} P_r g_{rd}}{\sigma^2(\sigma^2 + P_sg_{sr} + P_rg_{rd})})
\end{equation}

Assume there are two relays and let $a_1 = \frac{g_{sr_1}}{\sigma^2}$
and $b_1 = \frac{P_{r} g_{r_1 d}}{\sigma^2}$ similarily $a_2, b_2$ for relay 2.
At a particular source power $P_s$, relay 1 is chosen over relay 2
if $\frac{a_1b_1}{1+P_s a_1 + b_1} > \frac{a_2b_2}{1+P_s a_2 + b_2}
 $
\\
Consider the function $f(p) = \frac{pab}{1+p a + b} $ \\
\begin{equation}
f'(p) = \frac{(1+b)ab}{(1+pa+b)^2}
\end{equation}
$f'(p)$ is positive and decreasing with $p$. \\
The power at which both the relays give same rate can be obtained 
by equating $f_1(p)$ and $f_2(p)$.

\begin{equation}
	P_0 = (1+b_1)(1+b_2) \frac{\frac{a_1b_1}{1+b_1} - 
	\frac{a_2b_2}{1+b_2}}{a_1a_2(b_2-b_1)}
\end{equation}

For $P_0$ to be positive, both numerator and denominator should 
have same sign i.e., if 
$\frac{a_1b_1}{1+b_1} >	\frac{a_2b_2}{1+b_2}$ then $b_2 > b_1$.
	To explain this intuitively, let us assume $b_0,b_2$ to be 
	much larger than 1 which reduces the first inequialitely ti
	$a_1 > a_2$. What this means is, source to relay cahennel is 
	better for relay 1 but relay to destination channel is stronger
	for relay 2. Hence at low source powers relay 1 might givr
	better SNR but if we increase source power past $P_0$ relay 2 
	gives more rate than relay 1. Same arguments can be made 
	for the case where inequalities are in the opposite
	direction. 
	\\
	For $P_0 < 0$, one of the rel;ays is the desired one irrespec
	tve of source pwer.
\end{document}
